\documentclass[oneeqnum]{siamart220329}
\usepackage[T1]{fontenc}
\usepackage[utf8]{inputenc}
\usepackage{mathtools}

\newcommand*{\ldblbrace}{\{\mskip-5mu\{}
\newcommand*{\rdblbrace}{\}\mskip-5mu\}}

\protected\def\mathp{$p$}
\title{%
  Technical report: Supporting software for article
  ``Quasi-optimal Discontinuous Galerkin discretisations of the \mathp-Dirichlet problem''
}
\author{%
    Jan Blechta\thanks{\email{blechta@karlin.mff.cuni.cz}}
    \and
    Alexei Gazca\thanks{\email{alexei.gazca@mathematik.uni-freiburg.de}}
}
\headers{Quasi-optimal DG}{J.~Blechta, A.~Gazca}


\begin{document}
\maketitle


\begin{abstract}
  We discuss \href{https://github.com/blechta/quasioptimal-dg-p-laplace}{an
  implementation} of the smoothing operator from
  \cite{veeser-zanotti-ii,veeser-zanotti-iii}.
\end{abstract}


\begin{keywords}
  smoothing operator, quasi-optimal, DG, $p$-Laplace
\end{keywords}


%\begin{MSCcodes}
%\end{MSCcodes}


\tableofcontents


\section{Notation}
%
We consider the action of operator $E_p = A_p - B_p(\operatorname{Id}{}-A_p)$ from
\cite{veeser-zanotti-ii,veeser-zanotti-iii} for the lowest-order case~$p=1$
on triangles:
%
\begin{align*}
  E_1 v
  & = A_1 v + B_1 (v - A_1 v) \\
  & = \sum_{z\in\mathcal{L}^{\mathrm{int}}_1} v|_{K_z}(z) \, \varphi^1_z
    + \sum_{F\in \mathcal{F}^{\mathrm{int}}} \Bigl( \int_F \ldblbrace v
        \rdblbrace - \sum_{z\in\mathcal{L}^F_1} v|_{K_z}(z) \, \int_F \varphi_z^1
        \Bigr) \, \hat\varphi_F,
\end{align*}
%
where $\mathcal{L}^{\mathrm{int}}_1$ are the interior nodes of the first-order
Lagrange space (i.e., evalutation at internal vertices),
$K_z$ is an arbitrary mesh element containing~$z$,
$\varphi^1_z$ denote the first-order Lagrange basis function such that $\varphi^1_z(z')=\delta_{zz'}$ for mesh vertices $z$, $z'$,
$\mathcal{F}^{\mathrm{int}}$ is the set of internal facets,
$\ldblbrace\cdot\rdblbrace$ stands for facet average,
$\mathcal{L}^F_1$ denotes the first-order Lagrange nodes on facet~$F$ (i.e., facet vertices),
and
$\hat\varphi_F$ is the facet bubble normalized as $\int_{F'} \hat\varphi_F = \delta_{FF'}$.

The facet bubble can be represented as
%
\begin{equation}
  \label{eq:facet_bubbles}
  \hat\varphi_F
  = \frac{6}{|F|} \varphi^1_{z_F^1} \varphi^1_{z_F^2}
  = \frac{3}{2|F|} \varphi_F,
\end{equation}
%
where $\varphi^1_{z_F^j}$, $j=1,2$, are the first-order Lagrange functions
associated with facet vertices $z_F^1$, $z_F^2$, and $\varphi_F$ is the nodal
facet bubble, i.e., $\varphi_F=1$ at the midpoint of~$F$.

The above-defined $E_1$ has in its range only functions vanishing on the
boundary and hence, in this formulation, it only applies to homogeneous
Dirichlet problems. The same holds for this report and the reference
implementation, which can be obtained at
\url{https://github.com/blechta/quasioptimal-dg-p-laplace}.


\section{Lowest-order Crouzeix--Raviart case}
%
A~linear functional $f$ assembled against smoothed Crouzeix--Raviart basis $E_1
\varphi^{\mathrm{CR}}_F$, $F\in\mathcal{F}^{\mathrm{int}}$, can be expressed as
a~linear combination of $\langle f, \varphi^1_z \rangle$,
$z\in\mathcal{L}^{\mathrm{int}}_1$, and $\langle f, \varphi_F \rangle$,
$F\in\mathcal{F}^{\mathrm{int}}$,
%
\begin{multline}
  \label{eq:rhs_cr}
  \langle f, E_1 \varphi^{\mathrm{CR}}_F \rangle
  \\
  = \sum_{z\in\mathcal{L}^{\mathrm{int}}_1} \varphi^{\mathrm{CR}}_F|_{K_z}(z)
    \, \langle f, \varphi^1_z \rangle
  + \sum_{F'\in \mathcal{F}^{\mathrm{int}}} \Bigl( \int_{F'} \ldblbrace
    \varphi^{\mathrm{CR}}_F \rdblbrace - \sum_{z\in\mathcal{L}^{F'}_1}
    \varphi^{\mathrm{CR}}_F|_{K_z}(z) \, \int_{F'} \varphi_z^1 \Bigr)
    \, \langle f, \hat\varphi_{F'} \rangle.
\end{multline}
%
We have
%
\begin{subequations}
\begin{gather}
  \label{eq:cr_1}
  \int_{F'} \ldblbrace \varphi^{\mathrm{CR}}_F \rdblbrace = \delta_{FF'} |F|, \\
  \label{eq:cr_2}
  \varphi^{\mathrm{CR}}_F|_{K_z}(z) = \begin{cases}
    1  & \text{if $F\subset\overline{K_z}$ and $z\in\overline{F}$,} \\
    -1 & \text{if $F\subset\overline{K_z}$ and $z$ is a~vertex adjacent to~$F$,} \\
    0  & \text{otherwise,}
  \end{cases} \\
  \shortintertext{and}
  \label{eq:cr_3}
  \int_{F'} \varphi^1_z = \tfrac12 |F'| \quad \text{if $z\in\overline F'$.}
\end{gather}
\end{subequations}
%
The first term in~\cref{eq:rhs_cr} results in the index sets and coefficients
\texttt{coeffs1} computed in \texttt{SmoothingOpVeeserZanottiCR.coeffs()}.
The second term gives \texttt{coeffs2}, where the coefficients \texttt{3/2},
\texttt{$\pm$3/4} result from
%
\begin{alignat*}{2}
  \int_{F'} \ldblbrace \varphi^{\mathrm{CR}}_F \rdblbrace
  \, \langle f, \hat\varphi_{F'} \rangle
  &= \frac32 \langle f, \varphi_{F'} \rangle
  &\qquad& \text{if $F'=F\in\mathcal{F}^{\mathrm{int}}$,} \\
  \int_{F'} \varphi_z^1
  \, \langle f, \hat\varphi_{F'} \rangle
  &= \frac34 \langle f, \varphi_{F'} \rangle
  &\qquad& \text{if $F'\in\mathcal{F}^{\mathrm{int}}$ and
                 $z\in\mathcal{L}^{F'}_1\cap\mathcal{L}^{\mathrm{int}}_1$,} \\
\end{alignat*}
%
where we have used~\cref{eq:cr_1,eq:cr_3,eq:facet_bubbles}. Recall that
Firedrake uses nodal bubbles $\varphi_F$ normalized to one at the facet
midpoint.


\section{Lowest-order DG case}
%
Consider DG basis members $\varphi_{K,z}$, for mesh element~$K$, and
$z\in\mathcal{L}^{\mathrm{int}}_1$,
%
\begin{align}
  \nonumber
  \varphi_{K,z}|_{K'}(z') &= \begin{cases}
    1 & \text{if $K=K'$, $z=z'$, and $z\in\mathcal{L}^{\mathrm{int}}_1$,} \\
    0 & \text{otherwise,}
  \end{cases}
  \intertext{which in particular implies that}
  \label{eq:dg_avg}
  \varphi_{K,z}|_{K_{z'}}(z') &= \begin{cases}
    1 & \text{if $z=z'$, $z\in\mathcal{L}^{\mathrm{int}}_1$, and $K_{z'}=K$,} \\
    0 & \text{otherwise.}
  \end{cases}
\end{align}
%
Recall that, for $z\in\mathcal{L}_1$, $K_z$ is a~uniquely given mesh element
containing~$z$ (chosen at random or of smallest index, etc.).

Similarly to \cref{eq:rhs_cr} we have
%
\begin{equation}
  \label{eq:rhs_dg}
  \begin{aligned}
    \langle f, E_1 \varphi_{K,z} \rangle
    &= \sum_{z'\in\mathcal{L}^{\mathrm{int}}_1} \varphi_{K,z}|_{K_{z'}}(z')
      \, \langle f, \varphi^1_{z'} \rangle
    \\
    &+ \sum_{F'\in \mathcal{F}^{\mathrm{int}}} \Bigl( \int_{F'} \ldblbrace
      \varphi_{K,z} \rdblbrace - \sum_{z'\in\mathcal{L}^{F'}_1}
      \varphi_{K,z}|_{K_{z'}}(z') \, \int_{F'} \varphi_{z'}^1 \Bigr)
      \, \langle f, \hat\varphi_{F'} \rangle.
  \end{aligned}
\end{equation}
%
The first term in~\cref{eq:rhs_dg} with~\cref{eq:dg_avg} explains
\texttt{coeffs1} in \texttt{SmoothingOpVeeserZanottiDG\allowbreak.coeffs()}.
It is
%
\begin{equation*}
  \int_{F'} \ldblbrace \varphi_{K,z} \rdblbrace = \frac14 |F'|
  \qquad \text{if $F'\subset K$ and
               $z\in\mathcal{L}^{F'}_1\cap\mathcal{L}^{\mathrm{int}}_1$,}
\end{equation*}
%
which explains together with~\cref{eq:cr_3} and~\cref{eq:facet_bubbles}
the values \texttt{3/8} and \texttt{-3/4} in \texttt{coeffs2}, which
correspond to the second and third term in~\cref{eq:rhs_dg}, respectively.


\begingroup
\raggedbottom
\interlinepenalty=10000
\bibliography{references}
\bibliographystyle{siamplain}
\endgroup


\end{document}
